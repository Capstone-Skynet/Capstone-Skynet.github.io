\paragraph{\underline{T.P.1: Power Usage Test}}

This test checks actual power usage of the system.

\textbf{Addresses}: NF.PR.1

\textbf{Procedure}:
\begin{enumerate}[noitemsep]
    \item Setup the system in the lab.
    \item Power the Raspberry Pi and the FPGA board using two power supplies.
    \item Boot the system and start the machine learning implementation.
    \item Record the power reading (voltage and current).
\end{enumerate}

\textbf{Verification}: 
Compare the peak power with the maximum power listed in the user manual. If the recorded peak power does not exceed the theoretical maximum power, the test passes. 

\textbf{Results}:
Raspberry Pi Current Draw:
Idle: 300mA
Streaming: 350mA
Stress Test: 500mA

De1-SoC (comparable to Zedboard) Current Draw:
Idle: 250 mA
Light Program: 400 mA

Note: Current was measured at the output of the 12V battery bank.
Result: Tests Passed


%

\paragraph{\underline{T.P.2: Battery Safety Test}}

This test verifies the functionality of the battery protection circuit.

\textbf{Addresses}: F.PR.4, F.PR.5, F.PR.6

\textbf{Procedure}:
\begin{enumerate}[noitemsep]
    \item Connect the battery protection circuit to a lab power supply.
    \item Load it with an appropriately sized resistor in order to simulate normal operating conditions.
    \item Raise the supply voltage and probe the output voltage in order to verify that our overvoltage protection works correctly.
    \item Connect the supply voltage in backwards in order to verify that the reverse polarity protection works correctly.
\end{enumerate}

\textbf{Verification}: 
If sparking or overheating occurs during the test, then the test has failed.

\paragraph{\underline{T.P.3: Computing Platform Battery Test}}

This is a battery life test and it also verifies the stability of the supply voltage when the battery is not fully charged.

\textbf{Addresses}: F.PR.1, F.PR.4

\textbf{Procedure}:
\begin{enumerate}[noitemsep]
    \item Fully charge the battery.
    \item Breakout the battery to connect the multimeter to its output port.
    \item Record the voltage reading of the fully charged battery. 
    \item Connect the PMB and PLB to the battery and power up.
    \item Launch two testing program on both boards and let it run for 30 minutes.
    \item Record the voltage reading and current reading every minute.
    \item Power off the PMB and PLB and record the after-test voltage reading and battery level.
\end{enumerate}

\textbf{Verification}: 
To ensure a minimum of 30 minutes of operation time, the after-test battery level must be at least 50\%. The voltage drop must be within 2V to ensure a stable power output to the PMB and PLB.
