\paragraph{\underline{T.P.1: Power Usage Test}}

This test checks actual power usage of the system.

\textbf{Addresses}: NF.PR.1

\textbf{Procedure}:
\begin{enumerate}[noitemsep]
    \item Setup the system in the lab.
    \item Power the Raspberry Pi and the FPGA board using two power supplies.
    \item Boot the system and start the machine learning implementation.
    \item Record the power reading (voltage and current).
\end{enumerate}

\textbf{Verification}: 
Compare the peak power with the maximum power listed in the user manual. If the recorded peak power does not exceed the theoretical maximum power, the test passes. 

%

\paragraph{\underline{T.P.2: Battery Safety Test}}

This test verifies the functionality of the battery protection circuit.

\textbf{Addresses}: F.PR.4, F.PR.5, F.PR.6

\textbf{Procedure}:
\begin{enumerate}[noitemsep]
    \item Connect the battery protection circuit to a lab power supply.
    \item Load it with an appropriately sized resistor in order to simulate normal operating conditions.
    \item Raise the supply voltage and probe the output voltage in order to verify that our overvoltage protection works correctly.
    \item Connect the supply voltage in backwards in order to verify that the reverse polarity protection works correctly.
\end{enumerate}

\textbf{Verification}: 
If sparking or overheating occurs during the test, then the test has failed.