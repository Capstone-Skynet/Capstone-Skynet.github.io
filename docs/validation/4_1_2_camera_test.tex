\paragraph{T.CM.1: Camera Test}

This test verifies the attributes of the video that captured by the camera.

\textbf{Addresses}: NF.CAM.3, NF.CAM.4

\textbf{Procedure}:
\begin{enumerate}[noitemsep]
    \item The Raspberry Pi stores a 10-second video sample.
    \item Verify the video attributes on the Raspberry Pi.
\end{enumerate}

\textbf{Verification}: 
Compare the resolution and frame rate of the stored video with the setting.

%

\paragraph{T.CM.2: Camera Focusing Test}

This test verifies the camera's auto focus functionality.

\textbf{Addresses}: NF.CAM.1, NF.CAM.2

\textbf{Procedure}:
\begin{enumerate}[noitemsep]
    \item Place a testing object 10 meters away from the camera.
    \item Clear objects that is between the camera and the testing object.
    \item Take a picture using the camera with default setting.
\end{enumerate}

\textbf{Verification}: 
The focus should be on the testing object, that is, the testing object should be easily identified in the image.

%

\paragraph{T.CM.3: Video Recording Test}

This test verifies the video captured by camera can be stored into a video file.

\textbf{Addresses}: F.CAM.2, NF.CAM.6

\textbf{Procedure}:
\begin{enumerate}[noitemsep]
    \item Turn on the camera.
    \item Start recording for two hours.
\end{enumerate}

\textbf{Verification}: 
Verify that the recorded video file can be played back.

%

\paragraph{T.CM.4: Various Light Condition Camera Test}

This test verifies the automatic exposure functionality of the camera.

\textbf{Addresses}: NF.CAM.5

\textbf{Procedure}:
\begin{enumerate}[noitemsep]
    \item Select 5 different time in a day with different light conditions.
    \item Recording a 1-minute video at each of the selecting time at the same place.
\end{enumerate}

\textbf{Verification}: 
Verify that the objects in the recorded video can be easily identified.