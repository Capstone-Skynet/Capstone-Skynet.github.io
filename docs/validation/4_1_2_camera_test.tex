\paragraph{\underline{T.CM.1: General Camera Test}}

This test verifies the attributes of the video that captured by the camera.

\textbf{Addresses}: NF.CAM.3, NF.CAM.4

\textbf{Procedure}:
\begin{enumerate}[noitemsep]
    \item The PMB stores a 10-second video sample (through normal operation).
    \item Verify the video attributes on the Raspberry Pi.
\end{enumerate}

\textbf{Verification}: 
Compare the resolution and frame rate of the stored video with the previously stored configuration (640x480px,  10 FPS).

%

\paragraph{\underline{T.CM.2: Camera Focus Test}}

This test verifies the camera's focus capability.

\textbf{Addresses}: NF.CAM.1, NF.CAM.2

\textbf{Procedure}:
\begin{enumerate}[noitemsep]
    \item Place a test object 10 meters away from the camera (with an unobstructed line of sight).
    \item Capture video using the camera (with default settings).
\end{enumerate}

\textbf{Verification}: 
The object should be focused (i.e. the testing object should be easily identified and clear).

%

\paragraph{\underline{T.CM.3: Video Recording Test}}

This test verifies that the video captured by the camera is properly archived.

\textbf{Addresses}: F.CAM.2, NF.CAM.6

\textbf{Procedure}:
\begin{enumerate}[noitemsep]
    \item Perform normal capture operations.
    \item Review footage.
\end{enumerate}

\textbf{Verification}: 
Verify that the recorded video file can be played back.

%

\paragraph{\underline{T.CM.4: Various Light Condition Camera Test}}

This test verifies the automatic exposure functionality of the camera.

\textbf{Addresses}: NF.CAM.5

\textbf{Procedure}:
\begin{enumerate}[noitemsep]
    \item Select 5 different times in a day (with differing light conditions).
    \item Record a 1-minute video at each of the selected times, at the same place.
\end{enumerate}

\textbf{Verification}: 
Verify that objects in the recorded video can easily be identified.