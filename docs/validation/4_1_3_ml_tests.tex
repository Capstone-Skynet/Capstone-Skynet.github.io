\paragraph{T.ML.1: Machine Learning Algorithm Latency and Throughput}

This test verifies the latency and throughput of the hardware accelerated machine learning impelemntation on the FPGA.

\textbf{Addresses}:  F.ML.1, F.ML.2, F.ML.3, NF.ML.1, NF.ML.2

\textbf{Procedure}:
\begin{enumerate}[noitemsep]
    \item Program the FPGA with the machine learning implementation and monitoring of its latency and throughput.
    \item Set up computing platform and base station for live video capture and processing.
    \item Execute the system for 5 minutes while logging the produced data.
\end{enumerate}

\textbf{Verification}: 
For this test to pass, the logged data should show a consistent latency of one second or less to fulfill requirement NF.ML.1 and also show a consistent throughput of at least 2 frames per second to fulfill requirement NF.ML2.

%

\paragraph{T.ML.4: Real-time and Pre-recorded Video Processing}

This test verifies the ability of the machine learning implemntation to process both live inputs from the camera dn test inputs from a pre-recorded file.

\textbf{Addresses}:  F.ML.1, F.ML.2, F.ML.4

\textbf{Procedure}:
\begin{enumerate}[noitemsep]
    \item Program the FPGA with the machine learning implementation.
    \item Set up computing platform and base station for live video capture and processing.
    \item Execute the system for 5 minutes while logging the produced data.
    \item Reconfigue the computing platform and base station for processing of any pre-recorded video.
    \item Load the pre-recorded video into the computing platform.
    \item Execute the system for 5 minutes while logging the produced data.
\end{enumerate}

\textbf{Verification}: 
For this test to pass, the metadata produced by the machine learning implementation for both the live video feed and the pre-recorded video should be similiar in its qualitative properties. For object detection this would mean a consistent accuracy and quality of the produced metadata.

%