\paragraph{\underline{T.DR.1: PID Tuning and Testing}}

This test and set of procedures ensure that the flight controller unit (FCU) onboard the multirotor RPAS is capable of controlling the drone quickly and accurately. We are using established and empirical methods for tuning the PIDs; we will not consider system modeling and root-locus analysis of the system as we cannot accurately treat the RPAS as a linear and time invariant (LTI) system due to too many non-linearities.

\textbf{Addresses}: F.DR.1, F.DR.2, NF.DR.2

\textbf{Procedure (Auto-Tune)}:
\begin{enumerate}[noitemsep]
    \item Connect RPAS FCU to a computer and use the flight management software to program the FCU such that one of the radio channels triggers a program routine to automatically tune the PIDs (Auto-Tune).
    \item Connect RPAS to battery power and engage in a hovered flight. Ensure that a radius of 10 meters around the RPAS is clear of people or objects.
    \item Use the radio transmitter to enable the corresponding channel programmed for the auto-tune function on the FCU.
    \item Wait for Auto-Tune to control the RPAS and measure its response. From here, the automatically calculated PID values will be applied.
    \item Safely land the RPAS on the ground and disconnect power for post-flight analysis and logging.
\end{enumerate}

\textit{If the above auto-tuning process does not yield a controllable RPAS with fast responsiveness but sufficient damping to minimize overshoot, then move on to the following procedure.}

\textbf{Procedure (Manual-Tuning)}:
\begin{enumerate}[noitemsep]
    \item Slowly and iteratively increase or decrease the P gain until the drone feels agile and responsive.
    \item Slowly and iteratively increase or decrese the I gain until the drone is responsive with just noticeable wobble or control overshoot.
    \item Slowly and iteratively increase or decrease the D gain until the wobble and overshoot is reasonably dampened.
    \item Safely land the RPAS on the ground and disconnect power for post-flight analysis and logging.
\end{enumerate}

\textbf{Verification}: The PID tuning and testing is considered successful if the RPAS is easily controllable with fast response times using the radio transmitter. When the control sticks on the radio transmitter are reset to their centre, the drone should level-off. Significant deviations would suggest that the system has changed (e.g. change in center-of-gravity (COG), different motors and propellers, etc.) and re-tuning is required. 

% 

\paragraph{\underline{T.DR.2: RPAS Maximum Takeoff Weight Test}}

This test obtains the maximum thrust output, minimum thrust-to-weight ratio, and maximum power output of the RPAS prototype.

\textbf{Addresses}:  NF.DR.1, NF.DR.13, NF.DR.14

\textbf{Procedure}:
\begin{enumerate}[noitemsep]
    \item Attach the RPAS prototype to a scale or force sensor and ensure that the RPAS cannot separate scale or force sensor.
    \item Attach multimeter to the RPAS power circuitry.
    \item Fix or constrain the RPAS to the ground by using a heavy weight, tape, or fasteners.
    \item Tare the scale or force-sensor such that the measure is reference to the initial conditions.
    \item Power on the drone and arm the motors.
    \item Apply maximum throttle from the radio transmitter.
    \item Obtain voltage and current measurements from the multimeter.
    \item Obtain force measurements from the scale or force sensor.
\end{enumerate}

\textbf{Verification}: 
To ensure that the RPAS has minimum manuverability, at maximum take-off weight, we should be using 80\% throttle. For this test to pass, the RPAS must achieve a maximum pull of 125\% (2.5 kg) of maximum takeoff weight. Since this is a maximum load test case, we verify the measured current and voltage to ensure NF.DR.12 and NF.DR.13 are checked.

% 

\paragraph{\underline{T.DR.3: RPAS Typical Takeoff Weight Test}}

The typical takeoff weight test quantifies the flight-time of the RPAS during a typical operation with significantly less massive payload.

\textbf{Addresses}: NF.DR.8, NF.DR.11

\textbf{Procedure}:
\begin{enumerate}[noitemsep]
    \item Attach voltage and current monitor to the RPAS power circuitry.
    \item Set up RPAS for test-flight in open area or large room.
    \item Power on the drone and arm the motors.
    \item Start timer.
    \item Apply reasonable throttle from the radio transmitter such that the RPAS is hovering 3.0 m above ground.
    \item Record the \%-throttle applied for hovering.
    \item Keep hovering until the RPAS battery reaches 10\% capacity.
    \item Stop timer when the RPAS lands; calculate flight time in minutes.
    \item Measure the total mass of the RPAS and payload, and compute the average power consumption using the data acquired from the voltage and current monitors. Calculate efficiency in g/W.
\end{enumerate}

\textbf{Verification}:
The measured flight time should be greater than 20 minutes with a payload mass of 500 g in order to satisfy the requirement NF.DR.8. The calculated efficiency must be greater than 5.0 g/W to satisfy the requirement NF.DR.11.

% 

\paragraph{\underline{T.DR.4: Flight Time / Endurance Test}}

The flight time or endurance test tests the maximum flight time under optimal weight for optimal efficiency.

\textbf{Addressses}: NF.DR.8, NF.DR.11

\textbf{Procedure}:
\begin{enumerate}[noitemsep]
    \item Ensure that the RPAS has no payload or other added mass (i.e. only flight-essential hardware is onboard the RPAS).
    \item Take off and hover 3 m above ground.
    \item Start timer once the RPAS is stable in flight.
    \item Keep the RPAS fixed in place; apply adjustments if needed.
    \item When the battery warning starts to go off (when the battery has less than 15\% capacity), begin descent and gently land on the ground.
    \item Stop timer and record flight duration. Calculate efficiency in g/W.
    \item Perform multiple trials with variable paramter such as battery capacity or hover altitude if needed.
\end{enumerate}

\textbf{Verification}:
The measured flight time should be greater than the measured flight time with payload. The calculated efficiency must meet requirement NF.DR.8 and NF.DR.11.

\textbf{}