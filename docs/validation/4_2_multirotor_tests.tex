\paragraph{T.DR.1: RPAS Maximum Takeoff Weight Test}

This test obtains the maximum thrust output, minimum thrust-to-weight ratio, and maximum power output of the RPAS prototype.

\textbf{Addresses}:  NF.DR.1, NF.DR.12, NF.DR.13

\textbf{Procedure}:
\begin{enumerate}[noitemsep]
    \item Attach the RPAS prototype to a scale or force sensor and ensure that the RPAS cannot separate scale or force sensor.
    \item Attach multimeter to the RPAS power circuitry.
    \item Fix or constrain the RPAS to the ground by using a heavy weight, tape, or fasteners.
    \item Tare the scale or force-sensor such that the measure is reference to the initial conditions.
    \item Power on the drone and arm the motors.
    \item Apply maximum throttle from the radio transmitter.
    \item Obtain voltage and current measurements from the multimeter.
    \item Obtain force measurements from the scale or force sensor.
\end{enumerate}

\textbf{Verification}: 
To ensure that the RPAS has minimum manuverability, at maximum take-off weight, we should be using 80\% throttle. For this test to pass, the RPAS must achieve a maximum pull of 125\% (2.5 kg) of maximum takeoff weight. Since this is a maximum load test case, we verify the measured current and voltage to ensure NF.DR.12 and NF.DR.13 are checked.

% 

\paragraph{T.DR.2: RPAS Typical Takeoff Weight Test}

The typical takeoff weight test quantifies the flight-time of the RPAS during a typical operation with significantly less massive payload.

\textbf{Addresses}: NF.DR.7, NF.DR.10

\textbf{Procedure}:
\begin{enumerate}[noitemsep]
    \item Attach voltage and current monitor to the RPAS power circuitry.
    \item Set up RPAS for test-flight in open area or large room.
    \item Power on the drone and arm the motors.
    \item Start timer.
    \item Apply reasonable throttle from the radio transmitter such that the RPAS is hovering 3.0 m above ground.
    \item Record the \%-throttle applied for hovering.
    \item Keep hovering until the RPAS battery reaches 10\% capacity.
    \item Stop timer when the RPAS lands; calculate flight time in minutes.
    \item Measure the total mass of the RPAS and payload, and compute the average power consumption using the data acquired from the voltage and current monitors. Calculate efficiency in g/W.
\end{enumerate}

\textbf{Verification}:
The measured flight time should be greater than 20 minutes with a payload mass of 500 g in order to satisfy the requirement NF.DR.7. The calculated efficiency must be greater than 5.0 g/W to satisfy the requirement NF.DR.10.