\paragraph{T.G.1: Video Format Verification}

This test verifies the attributes of the video.

\textbf{Addresses}: F.CM.1

\textbf{Procedure}:
\begin{enumerate}[noitemsep]
    \item The Raspberry Pi stores a 10-second video sample.
    \item Verify the video attributes on the Raspberry Pi.
\end{enumerate}

\textbf{Verification}: 
Compare the resolution and frame rate of the stored video with the setting.

%

\paragraph{T.G.2: Parallel Computing Test}

This test verifies the structure of the computing platform.

\textbf{Addresses}: F.CP.2

\textbf{Procedure}:
\begin{enumerate}[noitemsep]
    \item Gather the required information of the computing platform. This includes: clock frequency, function unit count, function unit latency, and the latency of the data transmission.
    \item Calculate the theoretical processing time required to process a 1-minute video sample.
    \item Prepare the video sample.
    \item Feed the video sample to the machine learning model.
    \item Record the actual processing time used.
\end{enumerate}

\textbf{Verification}: 
If the difference between the theoretical value and actual value is less than 10\%, the test is passed. 