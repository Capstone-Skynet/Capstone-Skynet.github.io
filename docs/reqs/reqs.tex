\documentclass[10pt,letterpaper]{article}
\usepackage{amsmath}
\usepackage{amsfonts}
\usepackage{amssymb}
\usepackage[english]{babel}
\usepackage{breakurl}
\usepackage[superscript]{cite}
\usepackage{fancyhdr}
\usepackage{float}
\usepackage[margin=1in]{geometry}
\usepackage{graphicx}
\usepackage{hyperref}
\usepackage[utf8]{inputenc}
\usepackage{makeidx}
\usepackage{multicol}
\usepackage{nth}
\usepackage{longtable}

% Use Helvetica
\usepackage[scaled]{helvet}
\renewcommand\familydefault{\sfdefault} 
\usepackage[T1]{fontenc}

\usepackage{setspace}
\usepackage{siunitx}
\usepackage{svg}
\usepackage{subcaption}
\usepackage{tikz}
\usepackage{titling}
\usepackage{url}
\usepackage{kantlipsum}

\usepackage{parskip} % Paragraph skip - adds extra lineskip spacing
\setlength{\parskip}{0.7\baselineskip plus 2pt}

% Custom definitions
\newcommand{\doctitle}{Requirements Specification}
\newcommand{\docsubtitle}{Multirotor Computing Platform with FPGA Hardware Acceleration for Machine Learning}

% Custom commands
\newcommand{\ts}{\textsubscript}	% Subscript command %

% Use hyphans to break up urls
\def\UrlBreaks{\do\/\do-}

% PDF and href setup
% Hyper ref
\hypersetup{
	colorlinks=true,
	citecolor=black,
	linkcolor=black,
	filecolor=black,
	urlcolor=blue,
	pdftitle={\@title},
	bookmarks=true
}
\urlstyle{same}

% Page headings
\pagestyle{fancy}
\fancyhead[L]{\MakeUppercase{CPEN/ELEC 491}}
\fancyhead[R]{\textbf{Team 109}}
\fancyfoot{}
\fancyfoot[C]{\thepage}

% No paragraph indent
\parindent 0ex

% Meta
\author{
	Deutsch, Peter &
	\textit{me@peterdeutsch.ca}
	\\
	He, Muchen &
	\textit{i@muchen.ca}
	\\
	Hsueh, Arthur &
	\textit{ah11962@outlook.com}
	\\
	Wang, Meng &
	\textit{wzfftxwd@gmail.com}
	\\
	Wilson, Ardell &
	\textit{ardellw96@gmail.com}
}
\title{\doctitle}
\date{\today}

\makeatletter
\renewcommand{\maketitle}{
	\bgroup
	\setlength{\parindent}{0pt}
	\begin{flushleft}
		% Top spacing
		\vspace*{1in}

		% Title
		\textbf{\Huge{\@title}}\\
		\hrulefill

		% Subtitle
		\textbf{\huge{\docsubtitle}}
		
		\vspace*{0.5in}

		% Course number and team
		\textbf{\Large{CPEN/ELEC 491 Capstone Team 109}}\\
		\hspace*{0.1cm}
		\begin{tabular}[h]{|ll}
			\@author
		\end{tabular}

		\vspace*{0.25in}

		\textbf{Website}: https://capstone-skynet.github.io

		\vfill

		% Logo
		\hspace*{-0.3cm}\includegraphics[scale=0.5]{../assets/ece_logo.pdf}

		% Date
		\large{\@date}
	\end{flushleft}
	\egroup
}
\makeatother

% Begin Document
\begin{document}

% Title Page
\begin{titlepage}
	\maketitle
\end{titlepage}

% Revision history
\thispagestyle{empty}
\section*{Revision History}

Revision history written here.

\begin{table}[H]
\begin{tabular}{llll}
\hline
Version \# & Initials & Release Date & Changes Made \\ \hline
0.0 & PD & 2019-10-11 & Initial skeleton of the document.\\
\hline
\end{tabular}
\end{table}

% Table of contents
\setcounter{secnumdepth}{3}
\tableofcontents
\thispagestyle{empty}

% Terms and Abbreviations
\thispagestyle{empty}

\section*{Terms and Abbreviations}

\begin{tabular}[h]{rp{0.75\linewidth}}
    \hline
    \textbf{Term} & \textbf{Definition}\\
    \hline

    ANN & Artificial Neural Network, or simply ``Neural Network'', is a data processing model modeled after neuron interactions. The process consists of forward propagation through the use of several matrix multiplications.\cite{ann}\\
    ASIC & Application-Specific Integrated Circuit.\\
    CNN & Convolutional Neural Networks are neural networks which are especially useful for image classification.\cite{cnn} \\
    ECE & Department of Electrical and Computer Engineering at the University of British Columbia.\\
    FPGA & Field-Programmable-Gate-Array, a ``programmable'' hardware unit that allows for ASIC-like performance with software-like turn-around time and flexibility.\\
    FPS & Frames Per Second.\\
    GPU & Graphics Processing Unit, a discrete piece of hardware designed to accelerate graphics-intensive or other parallel computing tasks.\\
	ICICS & Institute for Computing, Information, and Cognitive Systems at the University of British Columbia.\\
    LOS & Line-of-sight.\\
    ML & Machine Learning.\\
    Multirotor & An unmanned vehicle with multiple engines. \\
    OTS & Off-the-shelf, or commercially available/purchasable. \\
    PID / PID Controller & Proportion-Integral-Derivative controllers denote the most common control algorithm for precise and accurate movement in multirotor applications.\cite{pid}\\
    RNN & Recurrent Neural Networks are neural networks where the output depends on previous computations, effectively consisting of memory.\cite{rnn}\\
    RX & Receiver.\\
    TC & Transport Canada.\\
    TX & Transmitter.\\
    YOLO & You-Only-Look-Once is a fast ML algorithm which detect objects, but is unlike CNN or RNN.\cite{yolo}\cite{yolo-2}\\
     & \\

    \hline

\end{tabular}

Technical terms and abbreviations dictionary go here.


% List of figures and tables
\thispagestyle{empty}
\listoffigures
\listoftables
\newpage

% Set page and section counter
\setcounter{page}{1}

% Begin content
\section{About This Document}\label{section:about}
\subsection{Purpose}\label{section:about:purpose}
\subsection{Intended Audience}\label{section:about:audience}
\section{Context}\label{section:context}
\section{Domain}\label{section:domain}

\section{Goal}\label{section:goal}

\begin{itemize}
    \item Drone with machine learning computing platfrom and camera attached as the payload.
    \item Drone is self-stabilizing using its onboard flight controller.
    \item Drone is capable of receiving radio signals from the pilot.
    \item Drone has at least 50\% flight duration of that without the payload, approximately 10 minutes minimum.
    \item Total takeoff mass of the drone with payload is less than 25kg.
    \item Onboard FPGA capable of performing parallel matrix computation to speedup machine learning.
    \item FPGA processing latency is one second or less.
    \item Machine learning model implemented on the FPGA is capable of detecting pedestrians from image frames.
    \item TODO
\end{itemize}

\section{Functional Specifications}\label{section:func-spec}

TODO: break down

\textbf{Integration of FPGA/Computer Vision Hardware with a Drone}:
As the ultimate deliverable consists of a highly mobile computing platform,
the student team's utmost objective is the successful integration of computation hardware with the drone.
The equipment for capturing video data, performing processing, and transmitting digital data must be reasonably compact and easily deployable.

\textbf{Air-to-Ground Data Transmission}:
Critical to research and analytics, the drone payload must be able to transmit video sensor data, in addition to processed data
from the onboard hardware accelerator (FPGA). There must be a ground station, consisting of a receiver and a display device (such as a mobile phone or 
computer) which receives and decodes the transmission for display, providing the user with a real-time video stream overlaid with machine learning intents.


\textbf{Machine Learning Implementation on the FPGA}:

The hardware-accelerated computing platform must correctly interface with attached peripherals (such as the camera sensor, the data transmitter (DTX), and other necessary hardware) through the use of well-defined communication protocols, such as serial or PCI-E (whichever is suitable given bandwidth requirements). The hardware accelerator must feature a starter ML model, implementing existing ML structures/frameworks such as (but not limited to) CNNs, RNNs, or YOLO \cite{yolo}, being fully capable of basic computer vision tasks.

\section{Non-Functional Specifications}\label{section:non-func-spec}

\section{Constraints}\label{section:const}
TODO: breakdown

The most pressing constraints are of an administrative nature. The project described
herein is very large in scope --- the constrained 8-month development period 
will likely affect the team's maneuverability and ability to comprehensively mitigate risks. 
In addition, navigating regulatory hurdles regarding drone piloting and radio data transmission (set out by 
Transport Canada and Industry Canada, respectively) will incur additional time and resource constraints.

As this device is intended to be used as a development platform in the future, it is important that it is future-proofed for further modifications. As detailed in Section \ref{section:businessbackground}, FPGAs are
inherently constrained with regards to on-chip resources. An appropriately-sized FPGA will need to be selected, capable of
implementing the base-level ``glue-logic'' required for the platform to operate (for example, the camera interface and data transmission logic), while still
leaving enough available resources on the FPGA for the client to implement their own ML models after the project's delivery. 
This constraint is tightly coupled with the high level of legacy risk inherent in this project, as outlined in Section \ref{risk}.

The physical properties of the payload (FPGA, electronics, camera, and  DTX) is an additional constraint. Since the purpose of the project
is to validate the viability of mobile ML designs, the payload should be light, compact, and power efficient, such
that it can be carried on a drone for a reasonable flight time in order to carry out useful ML applications.
The weight of the payload (and its required batteries) affects total flight duration, maximum altitude, airspeed, and maneuverability. 

As the bandwidth of video transmission is limited by transmission frequency and the transmitter (DTX) itself, 
critical communication-related design decisions must be made early in the project.
A transmitter with slightly longer range would significantly increase the power consumption due to inverse-square law.
Moreover, the data throughput depends heavily on a successful implementation of the communication
protocols the student team decides to choose. Clever compression or encoding techniques might need to be
employed to enhance the throughput and achieve a reasonable result.



% Bibliography
\clearpage
\addcontentsline{toc}{section}{References}
\bibliographystyle{ieeetr}
\bibliography{references}

% Appendix (uncomment to enable appendix)
% \clearpage
% \appendix
% \section{Appendix name}\label{appendix:sample-appendix}
% Content here

\end{document}