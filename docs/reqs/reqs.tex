\documentclass[10pt,letterpaper]{article}
\usepackage{amsmath}
\usepackage{amsfonts}
\usepackage{amssymb}
\usepackage[english]{babel}
\usepackage{breakurl}
\usepackage[superscript]{cite}
\usepackage{fancyhdr}
\usepackage{float}
\usepackage[margin=1in]{geometry}
\usepackage{graphicx}
\usepackage{hyperref}
\usepackage[utf8]{inputenc}
\usepackage{makeidx}
\usepackage{multicol}
\usepackage{nth}
\usepackage{longtable}

% Use Helvetica
\usepackage[scaled]{helvet}
\renewcommand\familydefault{\sfdefault} 
\usepackage[T1]{fontenc}

\usepackage{setspace}
\usepackage{siunitx}
\usepackage{svg}
\usepackage{subcaption}
\usepackage{tikz}
\usepackage{titling}
\usepackage{url}
\usepackage{kantlipsum}

\usepackage{parskip} % Paragraph skip - adds extra lineskip spacing
\setlength{\parskip}{0.7\baselineskip plus 2pt}

% Custom definitions
\newcommand{\doctitle}{Requirements Specification}
\newcommand{\docsubtitle}{Multirotor Computing Platform with FPGA Hardware Acceleration for Machine Learning}

% Custom commands
\newcommand{\ts}{\textsubscript}	% Subscript command %

% Use hyphans to break up urls
\def\UrlBreaks{\do\/\do-}

% PDF and href setup
% Hyper ref
\hypersetup{
	colorlinks=true,
	citecolor=black,
	linkcolor=black,
	filecolor=black,
	urlcolor=blue,
	pdftitle={\@title},
	bookmarks=true
}
\urlstyle{same}

% Page headings
\pagestyle{fancy}
\fancyhead[L]{\MakeUppercase{CPEN/ELEC 491}}
\fancyhead[R]{\textbf{Team 109}}
\fancyfoot{}
\fancyfoot[C]{\thepage}

% No paragraph indent
\parindent 0ex

% Meta
\author{
	Deutsch, Peter &
	\textit{me@peterdeutsch.ca}
	\\
	He, Muchen &
	\textit{i@muchen.ca}
	\\
	Hsueh, Arthur &
	\textit{ah11962@outlook.com}
	\\
	Wang, Meng &
	\textit{wzfftxwd@gmail.com}
	\\
	Wilson, Ardell &
	\textit{ardellw96@gmail.com}
}
\title{\doctitle}
\date{\today}

\makeatletter
\renewcommand{\maketitle}{
	\bgroup
	\setlength{\parindent}{0pt}
	\begin{flushleft}
		% Top spacing
		\vspace*{1in}

		% Title
		\textbf{\Huge{\@title}}\\
		\hrulefill

		% Subtitle
		\textbf{\huge{\docsubtitle}}
		
		\vspace*{0.5in}

		% Course number and team
		\textbf{\Large{CPEN/ELEC 491 Capstone Team 109}}\\
		\hspace*{0.1cm}
		\begin{tabular}[h]{|ll}
			\@author
		\end{tabular}

		\vspace*{0.25in}

		\textbf{Website}: https://capstone-skynet.github.io

		\vfill

		% Logo
		\hspace*{-0.3cm}\includegraphics[scale=0.5]{../assets/ece_logo.pdf}

		% Date
		\large{\@date}
	\end{flushleft}
	\egroup
}
\makeatother

% Begin Document
\begin{document}

% Title Page
\begin{titlepage}
	\maketitle
\end{titlepage}

% Revision history
\thispagestyle{empty}
\section*{Revision History}

Revision history written here.

\begin{table}[H]
\begin{tabular}{llll}
\hline
Version \# & Initials & Release Date & Changes Made \\ \hline
0.0 & PD & 2019-10-11 & Initial skeleton of the document.\\
\hline
\end{tabular}
\end{table}

% Table of contents
\setcounter{secnumdepth}{3}
\tableofcontents
\thispagestyle{empty}

% Terms and Abbreviations
\thispagestyle{empty}

\section*{Terms and Abbreviations}

\begin{tabular}[h]{rp{0.75\linewidth}}
    \hline
    \textbf{Term} & \textbf{Definition}\\
    \hline

    ANN & Artificial Neural Network, or simply ``Neural Network'', is a data processing model modeled after neuron interactions. The process consists of forward propagation through the use of several matrix multiplications.\cite{ann}\\
    ASIC & Application-Specific Integrated Circuit.\\
    CNN & Convolutional Neural Networks are neural networks which are especially useful for image classification.\cite{cnn} \\
    ECE & Department of Electrical and Computer Engineering at the University of British Columbia.\\
    FPGA & Field-Programmable-Gate-Array, a ``programmable'' hardware unit that allows for ASIC-like performance with software-like turn-around time and flexibility.\\
    FPS & Frames Per Second.\\
    GPU & Graphics Processing Unit, a discrete piece of hardware designed to accelerate graphics-intensive or other parallel computing tasks.\\
	ICICS & Institute for Computing, Information, and Cognitive Systems at the University of British Columbia.\\
    LOS & Line-of-sight.\\
    ML & Machine Learning.\\
    Multirotor & An unmanned vehicle with multiple engines. \\
    OTS & Off-the-shelf, or commercially available/purchasable. \\
    PID / PID Controller & Proportion-Integral-Derivative controllers denote the most common control algorithm for precise and accurate movement in multirotor applications.\cite{pid}\\
    RNN & Recurrent Neural Networks are neural networks where the output depends on previous computations, effectively consisting of memory.\cite{rnn}\\
    RX & Receiver.\\
    TC & Transport Canada.\\
    TX & Transmitter.\\
    YOLO & You-Only-Look-Once is a fast ML algorithm which detect objects, but is unlike CNN or RNN.\cite{yolo}\cite{yolo-2}\\
     & \\

    \hline

\end{tabular}

Technical terms and abbreviations dictionary go here.


% List of figures and tables
\thispagestyle{empty}
\listoffigures
\listoftables
\newpage

% Set page and section counter
\setcounter{page}{1}

% Begin content
\section{About This Document}\label{section:about}
This document will outline the project's requirements --- laying out the basis of a successful implementation.
\subsection{Purpose}\label{section:about:purpose}
\subsection{Intended Audiences}\label{section:about:audience}

\section{Functional Requirements}\label{section:goal}
\textit{This is a first draft, and is not yet a comprehensive list of requirements.}
\begin{itemize}
    \item Drone has machine learning computing platform and camera attached as the payload.
    \item Drone is self-stabilizing using its onboard flight controller.
    \item Drone is capable of being controlled by a pilot within line-of-sight.
    \item Total takeoff mass of the drone with payload is less than 25kg.
    \item Onboard FPGA is capable of performing parallel matrix computation to speedup machine learning.
    \item A machine learning model, implemented on the FPGA, is capable of detecting pedestrians from image frames captured using the camera.
    \item Computation platform transmits video data and ML model data to ground station.
    \item Ground station displays the video data overlaid with the ML model data.
    \item ML model data is presented in the form of bounding boxes.
\end{itemize}

\section{Non-Functional Requirements}\label{section:non-func-spec}
\textit{This is a first draft, and is not yet a comprehensive list of requirements.}
\begin{itemize}
	\item Drone has at least 50\% flight duration of that without the payload --- 10 minutes minimum.
	\item The drone must be able to fly in fair conditions (no rain, heavy winds, etc.)
	\item The machine learning model's processing latency is one second or less.
	\item The video displayed on the base station should have a resolution of at least 640x480 at 10FPS.
	\item Significant on-chip area (>50\%) should be dedicated for future client use.
\end{itemize}


% Bibliography
\clearpage
\addcontentsline{toc}{section}{References}
\bibliographystyle{ieeetr}
\bibliography{references}

% Appendix (uncomment to enable appendix)
% \clearpage
% \appendix
% \section{Appendix name}\label{appendix:sample-appendix}
% Content here

\end{document}