In late 2020, the client will undertake a research project assessing the feasibility of converting software-based machine-learning models into efficient hardware designs through the use of High-Level Synthesis (HLS) tools. The client would like to obtain a drone-mounted, FPGA-based demonstration platform in order to demonstrate the mobility and applications of their research. 

Although the idea of utilizing FPGAs on drones has been explored commercially\cite{existing-prod}, existing products focus on utilizing the FPGA for \textit{flight control} rather than implementing general purpose tasks (such as ML). To the best of the team's and client's knowledge, no general purpose drone-based FPGA platforms exist, nor can any existing commercial devices be retrofitted to implement ML tasks due to their restricted design scope.

In addition to the construction of the demonstration platform, the client would like an initial machine-learning application implemented on the device. This initial application will use a drone-mounted video camera to identify pedestrians through the use of machine learning, wirelessly transmitting the video and extracted pedestrian information to an external device (``base station''). The client will use this developed application as a starting point for further modifications/improvements stemming from their research. Additionally, the application could be used for a wide variety of purposes as-is, including applications towards disaster response, wildlife management, and demographic studies.

The student team selected to undertake this project is comprised of five students in the Capstone program offered by the Department of Electrical and Computer Engineering at the University of British Columbia.
