The main objectives are sorted by the client's priority in descending order.

\textbf{Integration of FPGA/Computer Vision Hardware with a Drone}:
As the ultimate deliverable consists of a highly mobile computing platform,
the student team's utmost objective is the successful integration of computation hardware with the drone.
The equipment for capturing video data, performing processing, and transmitting digital data must be reasonably compact and easily deployable.

\textbf{Air-to-Ground Data Transmission}:
Critical to research and analytics, the drone payload must be able to transmit video sensor data, in addition to processed data
from the onboard hardware accelerator (FPGA). There must be a ground station, consisting of a receiver and a display device (such as a mobile phone or 
computer) which receives and decodes the transmission for display, providing the user with a real-time video stream overlaid with machine learning intents.


\textbf{Machine Learning Implementation on the FPGA}:
The hardware-accelerated computing platform must correctly interface with attached peripherals (such as the camera sensor, the data transmitter (DTX), and other necessary hardware) through the use of well-defined communication protocols, such as serial or PCI-E (whichever is suitable given bandwidth requirements). The hardware accelerator must feature a starter ML model, implementing existing ML structures/frameworks such as (but not limited to) CNNs, RNNs, or YOLO \cite{yolo}, being fully capable of basic computer vision tasks.
