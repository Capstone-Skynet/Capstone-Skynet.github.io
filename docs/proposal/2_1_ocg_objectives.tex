The main objectives are sorted by the client's priority in descending order.

\textbf{Integration of Electronics with the Drone}:
Since the ultimate deliverable consists of a highly mobile computing platform,
our utmost objective is a successful integration of computation and processing hardware with the drone.
The equipment for capturing video data, performing processing, and transmitting digital data should be reasonably compact and easily deployable.
The thermal characteristics and total power consumption of the device should be reasonable such that they affect minimally
on the flight characteristics of the drone.

\textbf{Air-to-Ground Data Transmission}:
Critical to research and analytics, we need to ensure the drone payload is able to transmit video sensor data, as well as the processed data
from the onboard hardware accelerator (FPGA). There is a ground station, consisting of a receiver and a display device (such as a mobile phone or 
computer receives and decode the transmission for display, providing the client or user with real-time video stream and machine learning intents.
The wireless frequency and band chosen is compliant with local laws; no further actions are required if we utilize WiFi protocol (802.11n/ac in 2.4GHz or 5.2GHz)
using off-the-shelf WiFi modules.

\textbf{Machine Learning Implementation on the FPGA}:
The hardware-accelerated computing platform interfaces with attached peripherals such as the camera sensor, the data transmitter (DTX), and other hardware properly
using well-defined communication protocol, such as serial bus or PCI-E, whichever one that is suitable for the data bandwidth requirements. 
The FPGA should feature a starter ML model, such as but not limited to CNN, RNN, or YOLO, fully capable of basic computer vision.
The netlist design for the ML models are adequately synthesized with efficient datapath.
