The client of this project is Professor Mieszko Lis, a professor in the Electrical and Computer Engineering department at the University of British Columbia (UBC). A particular research focus of Dr. Lis is the acceleration of (traditionally software-based) machine-learning (ML) models through the use of dedicated hardware. 

The fabrication of custom-designed hardware is often prohibitively expensive - the manufacture of a single application-specific integrated controller (ASIC) can easily cost millions of dollars and take several years to complete. This is inherently unconducive to the quick and iterative prototyping required by small-scale research projects like those run by Dr. Lis. To mitigate this constraint, Dr. Lis would like to use Field Programmable Gate Arrays (FPGAs) - effectively programmable analogs of a real circuit - to prototype his ML hardware designs. Actual hardware designs can be (rapidly) modeled on an FPGA, which acts exactly like a traditional circuit, allowing for rapid iterative prototyping. Once a finalized design has emerged from the prototyping process, it can then be manufactured into an actual hardware chip, if desired.