The client of this project is Professor Mieszko Lis, a professor in the Electrical and Computer Engineering department at the University of British Columbia (UBC). One of the client's particular research foci is the acceleration of (traditionally software-based) machine learning (ML) models through the use of dedicated hardware.

The fabrication of custom-designed hardware is often prohibitively expensive --- the manufacturing of application-specific integrated circuits (ASICs) can easily cost millions of dollars and take several years to complete. This is inherently unconducive to the quick, iterative prototyping required by small-scale research operations like those run by Dr. Lis. To address this constraint, the client would like to use Field Programmable Gate Arrays (FPGAs) to prototype their ML hardware designs. FPGAs are effectively programmable analogs of ASICs -- trading off maximal circuit complexity and performance with a very short implementation time (a design can be implemented on an FPGA in a manner of minutes to hours) --- facilitating the rapid turnaround time required by the client. Once a finalized design has emerged from the prototyping process performed on the FPGA, the design can then be converted and implemented on an actual integrated controller, if desired.
