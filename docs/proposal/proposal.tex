\documentclass[10pt,letterpaper]{article}
\usepackage{amsmath}
\usepackage{amsfonts}
\usepackage{amssymb}
\usepackage[english]{babel}
\usepackage{breakurl}
\usepackage[superscript]{cite}
\usepackage{fancyhdr}
\usepackage{float}
\usepackage[margin=1in]{geometry}
\usepackage{graphicx}
\usepackage{hyperref}
\usepackage[utf8]{inputenc}
\usepackage{makeidx}
\usepackage{multicol}
\usepackage{nth}
\usepackage{longtable}

% Use Helvetica
\usepackage[scaled]{helvet}
\renewcommand\familydefault{\sfdefault} 
\usepackage[T1]{fontenc}

\usepackage{setspace}
\usepackage{siunitx}
\usepackage{svg}
\usepackage{subcaption}
\usepackage{tikz}
\usepackage{titling}
\usepackage{url}
\usepackage{kantlipsum}

\usepackage{parskip} % Paragraph skip - adds extra lineskip spacing
\setlength{\parskip}{0.7\baselineskip plus 2pt}

% Draft watermark (comment out to remove watermark)
% \usepackage{draftwatermark}
% \SetWatermarkText{\textsc{Draft}}
% \SetWatermarkScale{1}

% Custom definitions
\newcommand{\doctitle}{Project Proposal}
\newcommand{\docsubtitle}{Multirotor Computing Platform with FPGA Hardware Acceleration for Machine Learning}

% Custom commands
\newcommand{\ts}{\textsubscript}	% Subscript command %

% Use hyphans to break up urls
\def\UrlBreaks{\do\/\do-}

% PDF and href setup
% Hyper ref
\hypersetup{
	colorlinks=true,
	citecolor=black,
	linkcolor=black,
	filecolor=black,
	urlcolor=blue,
	pdftitle={\@title},
	bookmarks=true
}
\urlstyle{same}

% Page headings
\pagestyle{fancy}
\fancyhead[L]{\MakeUppercase{CPEN/ELEC 491}}
\fancyhead[R]{\textbf{Team 109}}
\fancyfoot{}
\fancyfoot[C]{\thepage}

% No paragraph indent
\parindent 0ex

% Meta
\author{
	Deutsch, Peter &
	\textit{me@peterdeutsch.ca}
	\\
	He, Muchen &
	\textit{i@muchen.ca}
	\\
	Hsueh, Arthur &
	\textit{ah11962@outlook.com}
	\\
	Wang, Meng &
	\textit{wzfftxwd@gmail.com}
	\\
	Wilson, Ardell &
	\textit{ardellw96@gmail.com}
}
\title{\doctitle}
\date{\today}

\makeatletter
\renewcommand{\maketitle}{
	\bgroup
	\setlength{\parindent}{0pt}
	\begin{flushleft}
		% Top spacing
		\vspace*{1in}

		% Title
		\textbf{\Huge{\@title}}\\
		\hrulefill

		% Subtitle
		\textbf{\huge{\docsubtitle}}
		
		\vspace*{0.5in}

		% Course number and team
		\textbf{\Large{CPEN/ELEC 491 Capstone Team 109}}\\
		\hspace*{0.1cm}
		\begin{tabular}[h]{|ll}
			\@author
		\end{tabular}

		\vspace*{0.25in}

		\textbf{Website}: https://capstone-skynet.github.io

		\vfill

		% Logo
		\hspace*{-0.3cm}\includegraphics[scale=0.5]{ece_logo.pdf}

		% Date
		\large{\@date}
	\end{flushleft}
	\egroup
}
\makeatother

% Begin Document
\begin{document}

% Title Page
\begin{titlepage}
	\maketitle
\end{titlepage}

% Executive Summary (Not used here in proposal)
% \include{executive_summary.tex}

% Revision history
\thispagestyle{empty}
\section*{Revision History}

Revision history written here.

\begin{table}[H]
\begin{tabular}{llll}
\hline
Version \# & Initials & Release Date & Changes Made \\ \hline
0.0 & PD & 2019-10-11 & Initial skeleton of the document.\\
\hline
\end{tabular}
\end{table}

% Table of contents
\setcounter{secnumdepth}{3}
\tableofcontents
\thispagestyle{empty}

% Terms and Abbreviations
\thispagestyle{empty}

\section*{Terms and Abbreviations}

\begin{tabular}[h]{rp{0.75\linewidth}}
    \hline
    \textbf{Term} & \textbf{Definition}\\
    \hline

    ANN & Artificial Neural Network, or simply ``Neural Network'', is a data processing model modeled after neuron interactions. The process consists of forward propagation through the use of several matrix multiplications.\cite{ann}\\
    ASIC & Application-Specific Integrated Circuit.\\
    CNN & Convolutional Neural Networks are neural networks which are especially useful for image classification.\cite{cnn} \\
    ECE & Department of Electrical and Computer Engineering at the University of British Columbia.\\
    FPGA & Field-Programmable-Gate-Array, a ``programmable'' hardware unit that allows for ASIC-like performance with software-like turn-around time and flexibility.\\
    FPS & Frames Per Second.\\
    GPU & Graphics Processing Unit, a discrete piece of hardware designed to accelerate graphics-intensive or other parallel computing tasks.\\
	ICICS & Institute for Computing, Information, and Cognitive Systems at the University of British Columbia.\\
    LOS & Line-of-sight.\\
    ML & Machine Learning.\\
    Multirotor & An unmanned vehicle with multiple engines. \\
    OTS & Off-the-shelf, or commercially available/purchasable. \\
    PID / PID Controller & Proportion-Integral-Derivative controllers denote the most common control algorithm for precise and accurate movement in multirotor applications.\cite{pid}\\
    RNN & Recurrent Neural Networks are neural networks where the output depends on previous computations, effectively consisting of memory.\cite{rnn}\\
    RX & Receiver.\\
    TC & Transport Canada.\\
    TX & Transmitter.\\
    YOLO & You-Only-Look-Once is a fast ML algorithm which detect objects, but is unlike CNN or RNN.\cite{yolo}\cite{yolo-2}\\
     & \\

    \hline

\end{tabular}

Technical terms and abbreviations dictionary go here.


% List of figures and tables
\thispagestyle{empty}
\listoffigures
\listoftables
\newpage

% Set page and section counter
\setcounter{page}{1}

\section{About This Document}\label{section:about}

\subsection{Purpose}

This document serves to facilitate a high-level understanding between the client and the student team 
regarding the goals and responsibilities of each party throughout the proposed Capstone project.
In addition to providing an overview of the project's background, objectives, and requirements, this 
document details how the project will be managed, provides a tentative project timeline, and documents the student team's risk mitigation strategies.

\subsection{Intended Audiences}

The client of the project is the primary audience of this document. The proposal is intended to present the client with a high-level overview of the project's implementation, both technical and administrative, as envisioned by the student team --- providing the client an opportunity to offer corrective feedback if required.

Secondary audiences of this document include the Capstone program's instructional team (serving as an update to the project's progress through Milestone I) and the authoring Capstone team itself (acting as a reference for for design decisions and project planning throughout the duration of the project).

\section{Background}\label{section:background}
This section outlines the context of the project. It will go over the background
of the suitor, and the project the suitor is proposing.

\subsection{Business Background}\label{section:businessbackground}
The client of this project is Professor Mieszko Lis, a professor in the Electrical and Computer Engineering department at the University of British Columbia (UBC). One of the client's particular research foci is the acceleration of (traditionally software-based) machine learning (ML) models through the use of dedicated hardware.

The fabrication of custom-designed hardware is often prohibitively expensive --- the manufacture of application-specific integrated circuits (ASICs) can easily cost millions of dollars and take several years to complete. This is inherently unconducive to the quick, iterative prototyping required by small-scale research operations like those run by the client. To mitigate this constraint, the client would like to use Field Programmable Gate Arrays (FPGAs) to prototype their ML hardware designs. FPGAs are effectively programmable analogs of ASICs -- trading off maximal circuit complexity and performance with a near-instant implementation time (a design can be implemented on an FPGA in a manner of minutes to hours) --- facilitating the rapid turnaround time required by the client. Once a finalized design has emerged from the prototyping process performed on the FPGA, the client may convert and implement the design on an actual integrated controller, if desired.


\subsection{Project Context}
The proposed project is an examination of the feasibility of  deploy trained machine learning model on board using field-programmable gate arrays(FPGAs) to the drone.

The ML implementation will focus on video processing for the object tracking component of this project.

The ground of the project is to expand on the application and possibilities for ML. FPGAs can have higher performance against conventional software and GPU applications. FPGAs also have lower power consumption and the capability of reconfiguration. This makes FPGAs a great option for a mobile computing platform. Implementing an FPGA for autonomous object tracking on a drone can streamline many potential processes, such as disaster response, wildlife management and demographic studies.

\section{Objectives, Goals, and Constraints}\label{section:ocg}
This section elaborates the objectives to be pursued in the project, the goals to be achieved, and constraints which might limit the project's scope and success.

The precise usage of the terms used throughout this section are as follows:

\begin{itemize}
    \item \textbf{Objectives}: The high-level requirements of the project, 
    all of which are required for the project to be deemed a success.
    
    \item \textbf{Goals}: The planned project specifications which will ultimately implement the project's objectives.
    
    \item \textbf{Constraints}: Factors which might alter or limit the execution of the project's goals and objectives.



\end{itemize}

\subsection{Objectives}
The main objectives are sorted by the client's priority in descending order.

\textbf{Integration of Electronics with the Drone}:
Since the ultimate deliverable consists of a highly mobile computing platform,
our uttermost most important objective is a successful integration of computation and processing hardware with the drone.
The equipment for capturing vidoe data, perform processing, and trasmitting digital data should be reasonbly compact and easily deployable.
the total power draw from the drone as well as power output, in terms of thermal output, should be reasonable such that they affect minimally
on the flight characteristics of the drone.

\textbf{Air-to-Ground Data Transmission}:
Critical to research and analytics, we need to ensure the drone payload is able to transmit video sensor data, as well as the processed data
from the onboard hardware accelerator (FPGA). There is a ground station, consist of a receiver and a display device such as a mobile phone or 
computer receives and decode the transmission for display, providing the client or user with real-time video stream and machine learning intents.
The wireless frequency and band chosen is compliant with local laws; no further actions are required if we utilize WiFi protocol (802.11n/ac in 2.4GHz or 5.2GHz)
using off-the-shelf WiFi modules.

\textbf{Machine Learning Implementation on the FPGA}:




\begin{itemize}
  \item The FPGA and its connected hardware receives usable video data.
  \item Adequately implemented datapath and or processor to facilitate data-flow.
  \item Adequately implemented machine learning model such as CNN, RNN, or YOLO in hardware.
  \item Video can be processed in real time with help of reduced frame rates and or resolution.
\end{itemize}

\subsection{Goals}
The main goal is by the end of the project, have a working implementation
of a ML computer vision model, capable of object detection and spacial
object tracking implemented on a commercial FPGA. The number of logic elements 
required to implement video processing, video transmission, and other
data tasks are within the hardware limit allowed by the budget.
The implementation is capable of being mounted on a drone and perform ML
tasks continuously and autonomously. Both the unprocessed (raw) and processed
data are to be transmitted wirelessly using 2.4GHz WiFi to a ground station, 
which could be a laptop computer, or a mobile phone. The transmitted video
data should have reasonable quality of at least 640x480 resolution and at 
a reasonable frame rate -- at least 10 fps. The total take-off weight of the
drone along with the processing hardware should not exceed 25 kilograms
(as specified by Transport Canada, a pilot with \textit{Basic Operations}
certificate or \textit{Advanced Operations} certificate cannot operate a 
drone heavier than 25kg). Lastly, the flight time of the drone is at least 
1 minute and can fly to an altitude of at least +10m to prove the viability
of the concept explored in this project.

\subsection{Constraints}
In pursuit of the objectives listed above, the main constraints relating 
each main objective is as follows:

The most important constraints are the non-technical ones. The project described
here is considered very large in scope. The limited 8-months period is extremely
limiting and thus will affect design decisions. The budget is a significant
constraint which will affect the components and parts the student team chooses.

The FPGA is a highly adaptive device that can deliver almost-ASIC level speed
but a large number of logic elements are required to implement a large model 
such as a machine learning - computer vision - model. The inherit constraints
with FPGA RTL designs such as timing, area, and power constraints will ultimately
limit the processing throughput of the video data. Bottleneck for limiting
processing resolution and frequency (frame rate).

Common machine learning models for computer vision is designed for GPU to
maximize parallel computing. The FPGA cannot match GPU data throughput, thus
the limited amount of logic elements would also constrain our architecture
design.

The physical form factor of the payload (FPGA, electronics, camera, and 
transmitter) is an important constraint. Since the purpose of the project
is to validate the viability, the payload should be light and compact such
that it can be carried on a drone for a reasonable flight time to carry
out applications.

The bandwidth of video transmission is also limited by transmission frequency.
The range of transmission is constrained by transmission power which is 
limited by power draw and heat.

\clearpage
\section{Project Plan}\label{section:project-plan}
This section outlines a preliminary responsibilities and tasks to be carried out.

\subsection{Final Milestone \& Ultimate Deliverables}\label{final_milestone}
This project concludes on \textbf{April 3rd, 2020}, at which point the client
and the instructors will receive the following project deliverables:

\subsubsection{Hardware Artifacts}

\begin{enumerate}

\item \textbf{Drone Prototype}:
A fully integrated FPGA computing platform mounted on a remote-controlled
drone that can capture video using an on-board camera.
The computing platform utilizes FPGA-based neural network accelerators to 
process the video data and detect and track one or more designated objects.
The platform transmits video data and associated machine learning metadata
to a ground station.

\item \textbf{Ground Station Prototype}:
A system that receives wireless video data and machine learning metadata from
the drone and displays both on-screen. The ground station also logs the received data 
to files for further research and analysis purposes.

\end{enumerate}

\subsubsection{Document Artifacts}\label{document-artifacts}

\begin{enumerate}

\item\textbf{Requirements Specification}:
A document outlining the functional and non-functional requirements
of the prototypes.

\item \textbf{Design Specification}:
A document describing the high-level architecture and design of technical subsystems.

\item \textbf{Validation Specification and Results}:
A document describing system testbenches, validation techniques, and validation/testing results.

\item \textbf{Operations, Maintenance, and Upgrades Specifications}:
A document, similar to a operation manual, outlining installation instructions,
recommended maintenance, and common troubleshooting guides.

\item \textbf{List of Deliverables}

\end{enumerate}

\subsubsection{Other Artifacts}

\begin{enumerate}

\item \textbf{Demonstrative Video}
\item \textbf{Oral Presentation and Poster}
\item \textbf{Project Repositories}: Repositories that include all source code,
generated netlists, CAD designs, spreadsheets, and other documents.

\end{enumerate}

\subsection{Intermediate Milestones}\label{intermediate_milestone}
There are three project milestones to track the project progress,
Milestones necessitate the delivery in-progress documents listed
in section \ref{document-artifacts}. Each milestone contains an oral
presentation which summarizes the project progress up to the milestone.

\subsubsection{Milestone I}
By milestone I (\textbf{October 15th, 2019}), the team finishes and submits
the \textit{Project Proposal} (\textit{this document}). The document outlines
the [baseline agreement among all stakeholders with 
regards to what is to be accomplished.]FIXME

\subsubsection{Milestone II}
Milestone II (\textbf{November 25th, 2019}) is the first prototype review. 
The review features initial progress in camera interface
and machine learning accelerator implementations on the FPGAs.
If the project progress is on track or ahead, we will demonstrate video-
capturing implementation synergizing with onboard ML model. Otherwise, we will
demonstrate these components functioning independently.

\subsubsection{Milestone III}
Milestone III (\textbf{February 10th, 2020}) is the second prototype review.
The review features improved ML accelerator and drone implementations. We will
focus on effort invested into video transmission, power supply circuitry, 
and batteries.

\subsection{Major Responsibilities}
This subsection covers the major responsibilities expected from the 
student team and the client.

\subsubsection{Team Responsibilities}
The student team is responsible for day-to-day development, management, and
operation of the project by coordinating group and client meetings. The
team is responsible for the final delivery of all artifacts listed in section
\ref{final_milestone}.
The team is expected to conduct research from academic or industry sources
for implementation methods, techniques, or processes.
Finally, the team is responsible for managing project finanace, inventory,
and acquisition of key material, electronics, and hardware.

\subsubsection{Client Responsibilities}
The client is responsible for being available to meet in-person or online
given a reasonable notice of one week. They are expected to provide
(reference to) necessary education or training material. They are also expected
to provide additional financial support if needed (see section \ref{budget} for
more detail).


\subsection{Schedule}
\subsubsection{Major Schedule}

\begin{figure}[!htb]
\centering
\newcommand{\ImageWidth}{12cm}

\usetikzlibrary{decorations.pathreplacing,positioning, arrows.meta}
\begin{tikzpicture}

% draw horizontal line   
\draw[thick, -Triangle] (0,0) -- (\ImageWidth,0) node[font=\scriptsize,below left=3pt and -8pt]{months};
\node[font=\scriptsize, text height=1.75ex,text depth=.5ex] at (10.7,-0.8) {years};

% draw vertical lines
\foreach \x in {0,1,...,8}
\draw (1.25 * \x cm,3pt) -- (1.25 * \x cm,-3pt);

\foreach \x/\descr in {0/9, 1.25/10, 2.5/11, 3.75/12, 5/1, 6.25/2, 7.5/3, 8.75/4, 10/5}
\node[font=\scriptsize, text height=1.75ex,
text depth=.5ex] at (\x,-.3) {$\descr$};

\node[font=\scriptsize, text height=1.75ex,text depth=.5ex] at (0,-0.8) {2019};
\node[font=\scriptsize, text height=1.75ex,text depth=.5ex] at (5,-0.8) {2020};

\newcommand\x{0.5}

\draw[green, line width=4pt] (0,\x) -- +(1.875,0);
\draw[blue, line width=4pt] (1.875,\x) -- +(1.6667,0);
\draw[red, line width=4pt] (3.541,\x) -- +(3.125,0);
\draw[-Triangle, dashed, red] (6.8,\x) --  +(3.2,0);

\draw [thick ,decorate,decoration={brace,amplitude=5pt}] (0,\x+.2)  -- +(1.86,0) 
       node [black,midway,above=4pt, font=\scriptsize] {Milestone I};
       
\draw [thick ,decorate,decoration={brace,amplitude=5pt}] (1.875,\x+.2)  -- +(1.65,0) 
node [black,midway,above=4pt, font=\scriptsize] {Milestone II};

\draw [thick ,decorate,decoration={brace,amplitude=5pt}] (3.541,\x+.2)  -- +(3.05,0) 
node [black,midway,above=4pt, font=\scriptsize] {Milestone III};

\draw [thick ,decorate,decoration={brace,amplitude=5pt}] (6.8,\x+.2)  -- +(3.15,0) 
node [black,midway,above=4pt, font=\scriptsize] {Final Prototype};



\draw[yellow, line width=4pt] (1.625,-1) -- +(1,0);
\draw[yellow, line width=4pt] (1.625,-2) -- +(1.5,0);
\draw[yellow, line width=4pt] (3.125,-3) -- +(1.25,0);

\draw [thick ,decorate,decoration={brace,amplitude=5pt}] (2.625,-1.2)  -- +(-1,0) 
       node [black,midway,below=4pt, font=\scriptsize] {Domestic Purchasing};
\draw [thick ,decorate,decoration={brace,amplitude=5pt}] (3.125,-2.2)  -- +(-1.5,0) 
       node [black,midway,below=4pt, font=\scriptsize] {International Purchasing};
\draw [thick ,decorate,decoration={brace,amplitude=5pt}] (4.375,-3.2)  -- +(-1.2,0) 
       node [black,midway,below=4pt, font=\scriptsize] {Drone Licensing and Certification};
       
\end{tikzpicture}
\caption{Project Timeline}\label{fig:timeline}
\end{figure}

The tentative timeline of the project is detailed in Figure \ref{fig:timeline}. Specific deliverables and goals for each milestone are detailed in sections \ref{final_milestone} and \ref{intermediate_milestone}.

\subsubsection{Weekly Schedule}

All team members have agreed to reserve the following time slots for group/client meetings, presentation preparations, and general developmental work:
\begin{itemize}
\item Tuesday, 15:00 - 17:30
\item Thursday, 15:00 - 17:30
\item Friday, 8:00 - 10:00
\end{itemize}

The team members also acknowledge that significant amounts of development effort will fall outside of these designated periods.

Team meetings are scheduled for Tuesdays at 15:30 in the Macleod Building, and meetings with the instructional team are currently scheduled for Thursdays at 16:00 in the Irving K. Barber Learning Centre. Meetings with the instructional team will transition to 15:30 on Tuesdays beginning in Term 2.

Formal client meetings will be held on an irregular (roughly weekly) basis, dependent on the client's availability. Peter Deutsch, the principal point of contact for the client, will provide additional project updates as needed via email, Skype, or informal in-person discussions.

\subsection{Budget}\label{budget}
The department provides a budget of \$650.

The following table lists the items required for this project.

\begin{center}
\begin{longtable}{|p{0.2\linewidth}|*2{>{\centering\arraybackslash}p{0.25\linewidth}|}}
\caption{Required items}\\
\hline
\textbf{Item} & \textbf{Estimated Price} & \textbf{Remark}\\
\hline
\endfirsthead
\multicolumn{3}{c}%
{\tablename\ \thetable\ -- \textit{Continued from previous page}} \\
\hline
\textbf{Item} & \textbf{Estimated Price} & \textbf{Remark}\\
\hline
\endhead
\hline \multicolumn{3}{r}{\textit{Continued on next page}} \\
\endfoot
\hline
\endlastfoot

Camera&\$30&\href{https://www.amazon.ca/Raspberry-Pi-Camera-Module-Megapixel/dp/B01ER2SKFS/ref=sr_1_3?crid=OOVX563QBZOF&keywords=raspberry+pi+camera&qid=1570511628&sprefix=raspbe\%2Caps\%2C238&sr=8-3}{Camera Module}\\ \hline
Transceiver&\$10&\href{https://leeselectronic.com/en/product/15897.html}{2.4G Transceiver}\\ \hline
Drone&\$150&\href{https://www.amazon.ca/JJRC-Quadcopter-Anti-Shake-Adjustable-Beginners/dp/B07T474NRG/ref=sr_1_1_sspa?keywords=drone&qid=1570513548&sr=8-1-spons&psc=1&spLa=ZW5jcnlwdGVkUXVhbGlmaWVyPUEyNEFHTDNSVEQ5S1c2JmVuY3J5cHRlZElkPUEwNjkxMDg1MVlQWFhIVVRVRVE0WiZlbmNyeXB0ZWRBZElkPUEwNjgzMjM3TU5aQ1NYV1RDNzY3JndpZGdldE5hbWU9c3BfYXRmJmFjdGlvbj1jbGlja1JlZGlyZWN0JmRvTm90TG9nQ2xpY2s9dHJ1ZQ==}{JJRC H68 Drone}\\ \hline
FPGA&\$300&Price estimated based on the price of the DE1 board\\
\end{longtable}
\end{center}

\subsection{Risk Profile}\label{risk}
\subsubsection{Risk Management}
Risks can have big influences on the whole project. It is important that we have an effective protocol to handle the potential risks.

The team has generated a list of potential risks\footnote{see section 4.7.2} and assign the likelihood and impact value to each of them according to the purpose and objective of the project. If the risk number, which is the product of likelihood and impact, of a risk is higher than 0.4, then it has a specific post-risk protocol. All the risks will be monitored and if there exits any signal of a risk, corresponding operations will be performed to prevent it from happening.

When a team member notices any signs of a risk, he will send a message to slack channel and add one card on Trello, the project management tool that the team is using, to alert the whole team. The card on Trello board will have a label showing the significance of the risk and the whole team will discuss about the operations that need to be performed.

The risk profile will be updated whenever the situation is changed, i.e. new potential risks emerged. 


\subsubsection{Risk Table}


\begin{center}
\begin{longtable}{|p{0.5\linewidth}|*3{>{\centering\arraybackslash}p{0.1\linewidth}|}}
\caption{Risk profile}\\
\hline
\textbf{Risk description} & \textbf{Likelihood} & \textbf{Impact} & \textbf{Risk} \\
\hline
\endfirsthead
\multicolumn{4}{c}%
{\tablename\ \thetable\ -- \textit{Continued from previous page}} \\
\hline
\textbf{Risk description} & \textbf{Likelihood} & \textbf{Impact} & \textbf{Risk} \\
\hline
\endhead
\hline \multicolumn{4}{r}{\textit{Continued on next page}} \\
\endfoot
\hline
\endlastfoot
FPGA logic capacity is not enough to implement our solution.&0.5&1&0.5\\ \hline
Electronics and battery are too heavy for the drone to take off.&0.5&1&0.5\\ \hline
Drone cannot take off at all due to drone hardware failure.&0.5&1&0.5\\ \hline
Accidents that damage to drone and FPGA that leads to extra budget that we may not have.&0.7&0.7&0.49\\ \hline
Forced to buy cheaper components which leads to lower performance.&0.8&0.5&0.4\\ \hline
Deliveries is not meeting client’s expectations&0.2&0.8&0.4\\ \hline
Knowledge and skill (Machine Learning and Computer vision) required for project is not sufficient.&0.5&0.8&0.4\\ \hline
Our scrum method is not efficient&0.5&0.7&0.35\\ \hline
Scope of the project is underestimated which leads to burn outs&0.5&0.7&0.35\\ \hline
Not have enough money.&0.5&0.7&0.35\\ \hline
Not enough time commitment from team member due to other courses.&0.5&0.6&0.3\\ \hline
Team is indecisive which leads to delay - management is not good enough to make a timely decision.&0.4&0.7&0.28\\ \hline
Team has lack of communication and make assumptions which leads to incompatibility&0.4&0.7&0.28\\ \hline
Documentation is not specific / complete.&0.7&0.4&0.28\\ \hline
Not enough time to work on documentation.&0.7&0.4&0.28\\ \hline
Scope of the project is overestimated which leads to diversification of vision of the project.&0.5&0.5&0.25\\ \hline
Not enough machine learning training data.&0.5&0.5&0.25\\ \hline
Illegal to fly drone.&0.3&0.8&0.24\\ \hline
New technology or research that changes the scope significantly&0.3&0.7&0.21\\ \hline
Technical debt paydown impact project timeline.&0.4&0.5&0.2\\ \hline
Software license does not allow our application to be delivered.&0.3&0.6&0.18\\ \hline
Key hardware components not available&0.2&0.8&0.16\\ \hline
Machine learning resources lack documentation&0.3&0.5&0.15\\ \hline
Camera lack documentation&0.3&0.5&0.15\\ \hline
FPGA lack documentation&0.3&0.5&0.15\\ \hline
Delivered documentation not adequate for usage and upgrade.&0.5&0.3&0.15\\ \hline
Selected software methodology creates issues with team member that decreases productivity&0.2&0.7&0.14\\ \hline
Legal changes significantly affect project - drones in particular&0.2&0.7&0.14\\ \hline
The final product cannot be maintained or extended.&0.7&0.2&0.14\\ \hline
Client is not cooperative or does not provide necessary information&0.2&0.7&0.14\\ \hline
Client wants to modify scope and requirements that leads to delays or cut features.&0.2&0.6&0.12\\ \hline
Development environment is inadequate.&0.2&0.6&0.12\\ \hline
Sabotage&0.1&1&0.1\\ \hline
Loss of client&0.1&1&0.1\\ \hline
Loss of team member&0.1&0.9&0.09\\ \hline
Equipment shipment is delayed or lost.&0.1&0.8&0.08\\ \hline
Client is not available enough to provide significant help.&0.1&0.5&0.05\\ \hline
Competitive offerings affect project requirement&0.1&0.5&0.05\\ \hline
Camera doesn’t work with FPGA&0.1&0.5&0.05\\ \hline
FPGA doesn’t work with transmission systems&0.1&0.5&0.05\\ \hline
\end{longtable}
\end{center}


\clearpage
\section{Approval}
\input{4_0_approval}

The client and the capstone team have agreed on the proposal.

\subsection*{Client}
\newcommand*{\SignatureAndDate}[1]{
    \par\noindent\makebox[3in]{\hrulefill} \hfill\makebox[2in]{\hrulefill}
    \noindent\makebox[3in][l]{#1}          \hfill\makebox[1.9in][l]{Date (YYYY-MM-DD)}
}

\vspace{0.5in}
\SignatureAndDate{Mieszko Lis}
\vspace{1in}

\subsection*{Capstone Team}
\vspace{.5in}
\SignatureAndDate{Peter Deutsch}
\vspace{.5in}
\SignatureAndDate{Muchen He}
\vspace{.5in}
\SignatureAndDate{Arthur Hsueh}
\vspace{.5in}
\SignatureAndDate{Meng Wang}
\vspace{.5in}
\SignatureAndDate{Ardell Wilson}

% Bibliography and referneces
\clearpage
\addcontentsline{toc}{section}{References}
\bibliographystyle{ieeetr}
\bibliography{references}

% Appendix (uncomment to enable appendix)
% \clearpage
% \appendix
% \section{Appendix name}\label{appendix:sample-appendix}
% Content here

\end{document}
