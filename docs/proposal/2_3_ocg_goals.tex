The ultimate goal of this project is to develop a mobile computing platform which utilizes an FPGA to perform hardware-accelerated machine learning tasks and deploy it on an unmanned multirotor aerial system (drone). 

The drone should be piloted manually within the line-of-sight (LOS) of the pilot, using a ground-to-air transmitter (TX) in the form of an off-the-shelf radio controller and receiver combination. The drone should feature a flight controller capable of self-stabilization using well-tuned PIDs and,
with the payload attached, the drone's flight duration should be at least 50\% of that without the payload (typically 10--15 minutes).
The total takeoff mass of the integrated system should not exceed 25 kilograms --- as specified by Transport Canada, as a pilot with a \textit{Basic} or \textit{Advanced Operations} certificate cannot operate a drone heavier than 25kg.

The machine learning model should analyze the video stream from the onboard camera and detect pedestrians within the frame with near-human accuracy. The model should exploit the FPGA's inherent parallelism to accelerate key ML tasks such as matrix multiplication and convolution. The model should output screen-space location of the detected pedestrians, with this prediction having a tolerable update latency of one second or less. In addition, the base station should display \textit{bounding boxes} surrounding the model's pedestrians predictions, overlaying a video of reasonable quality (at least 640x480, 10fps).
