The most pressing constraints are of an administrative nature. The project described
herein is very large in scope --- the constrained 8-month development period 
will likely affect the team's maneuverability and ability to comprehensively mitigate risks. 
In addition, navigating regulatory hurdles regarding drone piloting and radio data transmission (set out by 
Transport Canada and Industry Canada, respectively) will incur additional time and resource constraints.

As this device is intended to be used as a development platform in the future, it is important that it is future-proofed for further modifications. As detailed in Section \ref{section:businessbackground}, FPGAs are
inherently constrained with regards to on-chip resources. An appropriately-sized FPGA will need to be selected, capable of
implementing the base-level ``glue-logic'' required for the platform to operate (for example, the camera interface and data transmission logic), while still
leaving enough available resources on the FPGA for the client to implement their own ML models after the project's delivery. 
This constraint is tightly coupled with the high level of legacy risk inherent in this project, as outlined in Section \ref{risk}.

The physical properties of the payload (FPGA, electronics, camera, and  DTX) is an additional constraint. Since the purpose of the project
is to validate the viability of mobile ML designs, the payload should be light, compact, and power efficient, such
that it can be carried on a drone for a reasonable flight time in order to carry out useful ML applications.
The weight of the payload (and its required batteries) affects total flight duration, maximum altitude, airspeed, and maneuverability. 

As the bandwidth of video transmission is limited by transmission frequency and the transmitter (DTX) itself, 
critical communication-related design decisions must be made early in the project.
A transmitter with slightly longer range would significantly increase the power consumption due to inverse-square law.
Moreover, the data throughput depends heavily on a successful implementation of the communication
protocols the student team decides to choose. Clever compression or encoding techniques might need to be
employed to enhance the throughput and achieve a reasonable result.

