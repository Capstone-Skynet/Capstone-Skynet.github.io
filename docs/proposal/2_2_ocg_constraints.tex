In pursuit of the objectives listed above, the main constraints relating 
each main objective is as follows:

The most important constraints are the non-technical ones. The project described
here is considered very large in scope. The limited 8-months period is extremely
limiting and thus will affect design decisions. The budget is a significant
constraint which will affect the components and parts the student team chooses.

The physical properties of the payload (FPGA, electronics, camera, and  DTX) is an important constraint. Since the purpose of the project
is to validate the viability, the payload should be light and compact such
that it can be carried on a drone for a reasonable flight time to carry
out applications. The weight of the payload affects total flight duration, maximum altitude, airspeed, and manueverbility. 
There is a maximum cutoff for weight such that adding more battery capacity would actually negatively affect flight duration.

The power consumption of the onboard electronics also affects drone performance.
The computation tasks done on the FPGA are relatively intense, drawing significant energy from the drone's battery. 
to the drone battery. The power emission is also a constraint, excess heat from the processing
hardware need to be dissipated; however this constraint is not too severe.

The bandwidth of video transmission is limited by transmission frequency, transmitter (DTX), and
the implementation to interface with the transmitter. A transmitter with slightly longer range would
significantly increase the power consumption due to inverse-square law \cite{wiki-inverse-square}.
Moreover, the data throughput depends heavily on a successful implementation of the communciation
protocols the student team decides to choose. Clever compression or encoding techniques may be
employed to enhance the throughput and achieve a reasonable result.

The FPGA is a highly adaptive device that can deliver almost-ASIC level speed
but a large number of logic elements are required to implement a large model 
such as a machine learning - computer vision - model. The inherit constraints
with FPGA RTL designs such as timing, area, and power constraints will ultimately
limit the processing throughput of the video data. Bottleneck for limiting
processing resolution and frequency (frame rate).

Common machine learning models for computer vision is designed for GPU to
maximize parallel computing. The FPGA cannot match GPU data throughput, thus
the limited amount of logic elements would also constrain our architecture
design.
