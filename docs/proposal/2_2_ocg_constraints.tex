In pursuit of the objectives listed above, the main constraints relating 
each main objective is as follows:

The most important constraints are the non-technical ones. The project described
here is considered very large in scope. The limited 8-months period is extremely
limiting and thus will affect design decisions. The budget is a significant
constraint which will affect the components and parts the student team chooses.

The FPGA is a highly adaptive device that can deliver almost-ASIC level speed
but a large number of logic elements are required to implement a large model 
such as a machine learning - computer vision - model. The inherit constraints
with FPGA RTL designs such as timing, area, and power constraints will ultimately
limit the processing throughput of the video data. Bottleneck for limiting
processing resolution and frequency (frame rate).

Common machine learning models for computer vision is designed for GPU to
maximize parallel computing. The FPGA cannot match GPU data throughput, thus
the limited amount of logic elements would also constrain our architecture
design.

The physical form factor of the payload (FPGA, electronics, camera, and 
transmitter) is an important constraint. Since the purpose of the project
is to validate the viability, the payload should be light and compact such
that it can be carried on a drone for a reasonable flight time to carry
out applications.

The bandwidth of video transmission is also limited by transmission frequency.
The range of transmission is constrained by transmission power which is 
limited by power draw and heat.