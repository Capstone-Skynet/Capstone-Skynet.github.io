\subsubsection{Subsystem Overview}

The base station is a mobile device that connects to the computing platform on the multirotor. In particular, it performs the following tasks:
\begin{itemize}
    \item Displays the status of the system
    \item Displays the ML results in real-time (by overlaying the data over live video)
    \item Stores the log data and video for further analysis
    \item Remotely configures the computing platform (ex. starting object detection)
\end{itemize}

\subsubsection{Device Alternatives}
We considered two classes of devices which could be used as a base station: laptops and smart mobile devices (such as cellphones or tablets). The common features of these two options are: they are both equipped with a screen which can be used to display the video and ML results, they both have a Wi-Fi transceiver (\textbf{F.BS.3}), and they both have non-volatile storage which can be used to store results (\textbf{F.BS.6}).

One functionality that mobile devices lack is the support for common machine learning platforms (ex. TensorFlow). As the client may want to perform additional ML processing on the base station for further analysis in the future, laptops were ultimately selected as the base station's platform. Support for smart devices (to facilitate multiple viewers) may be added if time permits.

\subsubsection {Graphical User Interface}
A web-application is used as the GUI for the base station. This allows for universal compatibility with laptops and smart mobile devices, regardless of their operating system.

\subsubsection{Data Display and Function}
The base station hosts a TCP/IP server, and serves two clients: the PMB and the web-application. The server will continuously query the PMB for the live video feed data and ML results in the form metadata, and send the data to the web-application for display.

If an object detection scheme is implemented, the ML results will be displayed as a bounding box around the detected object(s) in the frame (\textbf{F.BS.1}). In particular, the ML results will consist of two parts: the coordinates of the upper-left vertex of the bounding box(es) and the height/width of the bounding box(es).

Buttons and radio buttons are situated on the web-application for user input (\textbf{NF.BS.2}). Their functions include adjusting the visual properties of the web-application and changing settings on the PMB and PLB (\textbf{F.BS.4})

The web application will also display real-time system status messages. These messages will come from the PMB or the base station itself. The status updates will include information on initiation or completion of processes, changes in PMB, PLB or base station settings and system warnings or errors.

All faults, either detected by the the computing platform or the base station itself will be logged locally on the device to facilitate further debugging, if necessary (\textbf{F.BS.5}).
%% maybe we just log all status messages, not limited to warnings/errors.

While the full system is operating, the base station will continuously save the received video and metadata to the user's device for any analysis purposes (\textbf{F.BS.6}).
