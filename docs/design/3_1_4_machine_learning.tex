% TODO: Describe potential machine learning applications. What could this look like, what will this \textit{not} look like? What kind of accelerators already exist? What might our solution look like?

% 

\subsubsection{Subsystem Overview} \label{ml_desc}
While machine learning is the topic that provides this project's context, this project is not focused on creating a machine learning model. Rather, the intention is to create a \textit{mobile computing platform} that \textit{allows} the client to perform research regarding hardware-accelerated machine-learning models.

As this is a mobile platform, the computing platform ultimately serves to take in physical data from the external world, process it, and output data which can be analyzed for the user's benefit. Examples of real-world situations which fit this category include disaster response, wildlife management, pedestrian detection, and demographic studies.

To provide a versatile demonstration of the platform's capabilities, the ubiquitous YOLOv2 machine learning model is used. Directly implemented by our selected open-source hardware accelerator (as described in Section \ref{ml_accel}), YOLOv2 facilitates the rapid bounding-box detection of multiple object types ('classes') such as pedestrians, bicycles, and animals. While the use of a less computationally expensive model (such as Tiny/MobileNet YOLO) would be preferred to allow for a higher framerate, the developmental effort to integrate such models into the existing hardware acceleration model is prohibitively excessive for the purposes of this Capstone project.

