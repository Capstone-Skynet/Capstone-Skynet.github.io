

\paragraph{Speed Controllers}

Electronic speed controllers (ESC) control the voltage, current, and phase supplied to a BLDC motor. As BLDC motors function by rotating magnetic fields using electric commutation, the ESCs must precisely control the timing and phase of the input. All ESCs function similarly and the design parameters are size/weight, power rating, power efficiency, and cost.

As shown in Table 2, the typical maximum current draw is approximately 30 A. For a 
safety margin of 10 A, we decided to select ESCs with maximum rating of 40 A. All of the following options are appealing:

\begin{table}[h]
    \centering
    \caption{ESC Purchasing Options}
    \label{table:esc-table}

    \begin{tabular}{lrrrll}

    \hline
    \textbf{ESC} & \textbf{Rating} & \textbf{Weight} & \textbf{Price}  & & \textbf{Vendor}\\
    & [A] & [g] & [CA\$] & & \\
    \hline
    HAKRC BLHeli Dshot1200 & 35 & 7  & 40.00 & per 4 & Banggood\\
    Racestart RS30A Lite & 30 & 6  & 40.00 & per 4 & Banggood\\
    Racestart SPROG X DShot600 & 35 & 4  & 30.00 & per 4 & Banggood\\
    Skystars Talon32 & 40 & 7  & 12.00 & per 1 & Banggood\\
    \hline

    \end{tabular} 
\end{table}

\subsubsection{Flight Controller Units}

Flight control units (FCUs) come in different tiers differentiated by their price and target application/market. 

The cheapest FCUs typically only have the bare-minimum features for flight (such as accelerometers and self-hover functions). These flight controllers are typically designed for acrobatic or basic VLOS flying and typically cost CA\$20 to CA\$50. The firmware that runs on these FCUs are typically rudimentary and open-source but nonetheless user-friendly and easily-programmable. Below is a list of FCs which falls under this tier:

\begin{itemize}[noitemsep,topsep=0pt, parsep=4pt, partopsep=0pt]
    \item F1, F3, F4, F7 FC are minimal in weight and footprint; they are ideal for light-weight operations.\cite{f1fc}
    \item KK 2.15 FC features a built-in display on the board, allowing quick access to flight settings without the need to connect to a computer for reprogramming.
    \item Naze32 FC is reliable with auto-tuning PIDs.
    \item CC3D FC is reliable with auto-tuning PIDs.
\end{itemize}

The next tier of FCs have more advanced features such as GPS-hold, autonomous flight, auto-land, telemetry, etc. They are designed for more advanced operations. These FCs are more expensive, costing CA\$50 to CA\$200, however these FCs have excellent after-sale support from their respective 
manufacturers and their documentation is better. Along 
with frequent manufacturer firmware updates, the mid-tier FCs are more reliable. The downside is that 
they are typically heavier and take up a much larger footprint. 
Below is a list of mid-tier FCs:

\begin{itemize}[noitemsep,topsep=0pt, parsep=4pt, partopsep=0pt]
    \item ArduPilot APM 2.8 is a flight controller using Arduino Mega and supports GPS and telemetry.
    \item Pixhawk PX4 Autopilot features pre-programmable autonomous operations.
    \item DJI Naza M Lite.
\end{itemize}

For the applications we require, we chose the ArduPilot APM 2.8 or the Pixhawk PX 4 as they are the most flexible options with advanced features at a reasonable price of CA\$75.00. Additionally, due to of the low-cost of low-tier FCs, we will also purchase one of the Naze32 or CC3D FC modules as a back-up.

\subsubsection{Radio Systems}

The minimum number of channels required to fly a drone is 4 -- one for each control: throttle, yaw, 
pitch, and roll. For this reason, we opted for the cheapest radio transmitter (TX) and receiver  (RX) 
combo we found from online vendors. 
The FlySky-FS i6 2.4GHz 6-Channel TX and RX bundle is ideal for our application due to its low cost.
According to the FlySky-FS i6 datasheet\cite{flyskyi6}, the radio frequency (RF) peak power is below 20 dBm while still achieving a maximum control range of 500 m (\textbf{NF.DR.4}).

\subsubsection{Battery}

We chose lithium polymer (Li-Po) batteries as they have the highest energy density (highest capacity to weight ratio), and are thus  perfect for high-power, low-weight applications. Figure~\ref{fig:batterytypes} shows a graph of the energy densities for differing types of batteries \cite{battery} --- from this graph, it is clear that Li-Po batteries are to be chosen.

\begin{figure}[h]
    \centering
    \includegraphics[width=0.5\textwidth]{img/energydensity.png}
    \caption{Energy Density Graph of Various Battery Types}
    \label{fig:batterytypes}
\end{figure}

Li-Po batteries have four main design parameters: configuration, capacity, discharge rating, and internal resistance.

\textit{Configuration} is how the Li-Po batteries are manufactured and wired together: a single cell in a Li-Po battery provides a 
nominal voltage of 3.7 V due to its electrochemistry.
Similar to how portable electronics and appliances require AA or AAA batteries placed in a certain order, a single cell of 3.7 V is not enough to drive anything powerful. Ergo multiple cells can be wired in series and the 
voltage adds up: 2 cells in series, denoted as ``2S'' batteries, have 7.4 V output, 3S batteries have 11.1 V output, and 4S batteries have a 14.8 V output. 

Battery \textit{capacity} is measured in watt-hours (Wh) or milli-amp-hours (mAh), which are units of electric power multiplied by units of time. A 20.0 Wh battery can output 20.0 W for 1.0 hour. Multiple battery cells can be put in parallel and their capacity adds up. A pair of four-cell batteries wired in parallel, denoted by ``4S2P'' has a 14.8 V output while having double the capacity of a single 4S battery of the same variant.

The \textit{discharge rating} is denoted by ``C'' and indicates the ratio of the capacity and the maximum discharge current which can be safely discharged without significantly harming the battery's life. It is important to note that a high discharge rating does not imply improved performance due to the presence of \textit{internal resistance}. A higher internal resistance implies that the battery can run less efficiently, as power is lost internally\cite{battery-c}. We will thus choose the battery with the lowest internal resistance as allowed by our budget. 

We will choose the design parameters for the battery after we have consolidated on the rest of the parts for the RPAS -- allowing for more system flexibility. 
