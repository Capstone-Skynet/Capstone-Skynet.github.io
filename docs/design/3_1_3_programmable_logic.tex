\subsubsection{Subsystem Overview}
 The PLB manages the hardware-accelerated ML application, performs additional pre/post-processing on ML data (if required by the ML application), and communicates with the platform management board.

A decomposition of the system is seen in Figure \ref{pcdiag}.

Of note is the distinction between the two portions present in the PLB: the Hard Processor System (HPS) and the FPGA. The Hard Processor System is a non-customizable CPU and is used for generic tasks such as managing I/O via the Ethernet port and processing non-acceleratable ML tasks. The FPGA provides the system's \textit{programmable circuity} -- performing the requisite hardware acceleration needed for ML tasks. The two systems co-reside on the same chip, and are interfaced via an internal bus.

\subsubsection{Device Selection}
The selected programmable logic board is the \textit{ZedBoard SoC Development Board} (pictured in Figure \ref{zedboard}), manufactured by Digilent. 

\begin{figure}
\centering
\includegraphics[width=12.5cm]{img/zedboard_functional_overview.jpg}
\caption[Layout of Connectors and Main ICs on ZedBoard]{Layout of Connectors and Main ICs on ZedBoard \cite{zedboard}}
\label{zedboard}
\end{figure}

The system specifications are as follows\cite{zedboard}:
\begin{itemize}
\item Xilinx Zynq-7000 AP SoC XC7Z020-CLG484
\item Dual-core ARM Cortex-A9 (\textit{HPS})
\item 512 MB DDR3 Random Access Memory
\item 256 MB Quad-SPI Flash Memory
\item 10/100/1000 Ethernet 
\item USB OTG 2.0 and USB-UART 
\item PS \& PL I/O expansion (FMC, Pmod, XADC)
\end{itemize}

The selection of this particular programmable logic board was mainly motivated by cost. The client was able to provide several ZedBoard units free-of-charge, stemming from donations through the Xilinx University Program. While this board has a fairly small number of logical cells (85,000 LCs) compared to other programmable logic boards on the market, especially those which target video applications, the project's budget does not allow for the purchase of more performant boards without significant sacrifices to the multirotor assembly. The number of logical cells on the ZedBoard is sufficient to comply with the minimum LC constraint (\textbf{C.EX.1}).

Examples of other boards examined for this project include the Nexys Artix 7 FPGA (200,000 LCs, US\$479)\cite{nexys} and the Microsemi PolarFire Video/Imaging FPGA (300,000 LCs, CAN\$1500)\cite{microsemi}. Use of the Terasic DE1-SoC (another board available free-of-charge from the client, 85,000 LCs) was also considered, but was ultimately not selected due to its poorer I/O selection and more complicated development toolchain.

\subsubsection{Data Flow}
The overall data flow managed by the PLB is as follows:
\begin{enumerate}
\item Video/ML input data is acquired by the HPS (ARM core) via the Ethernet port (from the PMB)
\item After performing any necessary clean-up/image resizing, the video frames are sent from the HPS to the FPGA via a bi-directional AXI interface
\item The ML IP block on the FPGA receives feature vectors (video frames) to be analyzed via the AXI bus, performs the requisite processing, and sends the results to the HPS (on the same bus)
\item The HPS interprets the results (ex. applying a sensitivity threshold) and sends these interpretations to the PMB (Raspberry Pi) via the Ethernet port
\end{enumerate}

\subsubsection{ML Hardware Accelerator Implementation}\label{ml_accel}
To facilitate the hardware acceleration of ML subtasks, a Zynq-7000-based, open-source hardware accelerator \cite{yolov2accel} (developed by graduate students at Jiangnan University) is utilized on the FPGA. As described in the accelerator's associated 2019 whitepaper\cite{yolov2unipaper}, the accelerator is capable of performing YOLOv2 processing 112.9$\times$ faster and with 86$\times$ less power than a comparable implementation on an ARM A9 core (similar to the HPS on the Zedboard). In practice, this allows for a frame-rate (throughput) of approximately 1 FPS. As the client intends to replace this module with a custom-designed solution stemming from their research, however, the YOLOV2 accelerator is only intended as a placeholder.

\textbf{TODO: }Add quick explanation as to how the accelerator works

In addition to the YOLOV2 accelerator, two alternative hardware acceleration units were considered for the PLB: \textit{MARLANN}\cite{marlann} and \textit{FREE-TPU}\cite{freetpu}. 

\textit{MARLANN} is an open-source capable of performing 8-bit fixed point signed integer arithmetic, matrix multiplication, and max-pooling. While MARLANN has a low FPGA area overhead (approx. 5,000 LCs), its relative simplicity would necessitate a large amount of effort to develop glue logic to develop a functional ML system. The MARLANN project is also not regularly maintained, leading to further concerns regarding future project maintenance/extensibility.

\textit{FREE-TPU} is a free (MIT licensed) acceleration platform developed by EmbedEEP. While the platform allows for greater flexibility in model selection compared to our selected solution (allowing for more appropriate models such as MobileNet-YOLO), the project is no longer maintained. In addition, as the source code is unavailable (the API is made available as a shared library), we are prohibited from making appropriate modifications to better suit our platform (ex. image data structures) leading to unacceptably poor performance (less than 0.25 FPS).

 
