\subsubsection{Subsystem Overview}
The platform management board manages all \textit{computing} I/O required by the multirotor: interfacing with the camera, programmable logic board (hosting the machine learning implementation), and the base-station. A separate system is used for the multirotor's flight control.

A decomposition of the system is seen on the left in Figure \ref{pcdiag}.

\begin{figure}\label{rpi}
\centering
\includegraphics[width=10cm]{img/RaspberryPi_Model_4B.png}
\caption{Layout of Connectors and Main ICs on Raspberry Pi 4}
\end{figure}

\subsubsection{Device Selection}
The selected device to implement the platform management system is the Raspberry Pi 4 (Model B). As seen in Figure \ref{rpi}, the Raspberry Pi 4 is a small single-board computer which hosts an ARM Cortex A72 CPU, a CSI camera interface, 40 serial pins, WiFi/Bluetooth, and a (1 gigabit) ethernet port. The Raspberry Pi 4 was selected over other single-computer platforms (such as Arduino/Beagleboard) as it:
\begin{itemize}
\item is relatively inexpensive (CAN\$45 in November 2019),
\item has a large support community and a wealth of driver/library support,
\item supports a wide range of high throughput I/O standards (USB/WiFi/Ethernet/Serial, \textbf{F.CP.3}),
\item provides an industry-standard camera interface (\textbf{F.CAM.2}), and
\item supports extensible non-volatile storage (\textbf{F.CP.2})
\end{itemize} 

\subsubsection{Data Flow}
The overall data flow managed by the Raspberry Pi is as follows:
\begin{enumerate}
\item Video is acquired through the camera CSI interface
\item Frame manipulation/correction is performed on the video, if required by the specific machine learning use-case
\item Manipulated video is decomposed and transmitted over Ethernet to the programmable logic board (which performs the machine learning)
\item Machine learning results are received via Ethernet
\item Video, ML results, and system status information are transmitted via WiFi to the base station
\item (Optional) Results and video are stored locally on non-volatile storage for post-flight analysis
\end{enumerate}

All operations are managed by the ARM Cortex A72 chip running a headless version of the Raspbian operating system. 

\subsubsection{Single-Board vs Two-Board Solution}
The presence of on-board WiFi/Camera interfaces is the primary driving factor towards a two-board (Platform Management and Programmable) solution. While a single-board solution is possible, the development effort and cost of the computing platform would be considerably higher. Most programmable logic boards (including our chosen board, the Zedboard) require external expansion devices to interface with peripherals such as WiFi/cameras. These expansion devices, which connect to the programmable logic board via a serial connection, have poorly documented drivers and are of considerable expense -- ranging from US\$25 for a basic WiFi module to US\$200 for a compatible camera interface card. External expansion devices also require considerable glue logic resources on the programmable logic board, potentially risking non-compliance with constraint \textbf{C.EX.2}.

An additional consideration in selecting the two-board design regards platform extensibility --- the client may desire to increase the programmable logic capabilities of the computing platform in the future, requiring a board swap. If a single-board solution is implemented, the client would be required to migrate \textit{all} hardware/software to the larger board, requiring them to devote considerable effort to debug driver issues. Additionally, as the external expansion devices are not necessarily compatible with all PLBs --- adding constraints to the client's board selection or requiring the purchase of new interface cards. By offloading the interfacing work to a distinct \textit{platform management} board, the client can easily replace the PLB without considerable redevelopment (fulfilling constraint \textbf{C.EX.3}).

\subsubsection{Interface with Programmable Logic Board}

To interface with the PLB, data packets are sent via a bi-directional TCP/IP-managed Ethernet connection. TCP/IP over Ethernet was selected in lieu of serial solutions (such as I2C) as:
\begin{itemize}
\item Ethernet is equally as ubiquitous on modern programmable logic boards,
\item Ethernet has higher maximum throughput (1 Gbps on Ethernet vs. 400 Kbps on serial) (\textbf{F.CP.3}),
\item Ethernet has more resilient error correction capabilities. (\textbf{F.CP.1})
\end{itemize} 

The selection of Ethernet over serial precludes the use of the \textit{Zero W} variant of the Raspberry Pi, as the Zero W does not provide Ethernet support. While the Raspberry Pi 4 is considerably heavier (46 g for Raspberry Pi 4 vs. 9 g for Raspberry Pi Zero W), this difference is marginal when considering the total weight of the multirotor.
