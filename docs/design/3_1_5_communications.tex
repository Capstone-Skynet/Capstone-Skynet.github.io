\subsubsection{Subsystem Overview}
The Air-Ground communication scheme between the platform management board and the base station is responsible for transmitting video, ML results, and control commands. It is a bi-directional, high bandwidth communication channel that requires a stable connection range up to 30 meters away \textbf{NF.CM.1}.

There are two options for establishing this communication link - single channel or dual channel. 

The first option is to use the existing communication channel to the multirotor (used for drone control), to transmit data. This would be advantageous as it doesn't require additional hardware and saves power. The disadvantage is that the drone control is using an off-the-shelf module, and in order to extend its functionality we would need to make modifications to a highly integrated system. 

The second option is to use a separate air-ground communication channel. This channel can be established under commonly used wireless communication methods (ex. Wi-Fi or Bluetooth), requiring a dedicated pair of transceivers. The advantage of this option is that most development boards are equipped with wireless transceivers and the communication scheme is easily configured. As such, we decided to employ this option.

\subsubsection{Bandwidth Requirements}

The majority of the bandwidth will be reserved for video transmission as it's the most intensive transmission task performed by the computing platform. The total number of bits per second of video that need to be transmitted can be calculated using the following formula:

$$
\text{\# of bits/second} = \frac{\text{\# of bits}}{\text{1 pixel}} \times \frac{\text{\# of pixels}}{\text{1 frame}} \times \frac{\text{\# of frames}}{\text{1 second}}
$$

The video that will be transmitted, assuming no compression, has a resolution of 640x480 pixels at 30 frames per second. The colour depth is 24-bit for a RGB colour set (that is, each pixel is 8 bits). As such:

$$
\text{\# of bits/second} = \frac{8}{\text{1 pixel}} \times \frac{640\times 480}{\text{1 frame}} \times \frac{30}{\text{1 second}} = 73.738~\text{Mbits/s}
$$

The other data that needs to be transmitted includes the video metadata, the results from ML model, the control commands, and other protocol overheads. Thus, an upper bound for the total bandwidth required by the system is roughly 75 Mbps.

\subsubsection{Communication Scheme}

The communication method we chose for the project is Wi-Fi. The specific standard we chose to utilize is IEEE 802.11n (Wi-Fi 4), as it has enough throughput and nearly all modern devices equipped with Wi-Fi support this standard. (\textbf{C.EX.3})

The advantages of Wi-Fi over other communication methods (such as Bluetooth) are as follows:
\begin{itemize}
    \item Long transmission range. IEEE 802.11n Wi-Fi has a transmission range of up to 53 m. (\textbf{NF.CM.1})
    \item High transmission rate. IEEE 802.11n Wi-Fi has a transmission rate up to 600 Mbps.
    \item No license is required.
    \item After the initial configuration, the connection is capable of auto-establishment. (\textbf{F.CM.3})
\end{itemize}